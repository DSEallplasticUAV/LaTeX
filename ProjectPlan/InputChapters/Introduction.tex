\documentclass[a4paper]{report}
\usepackage[english]{babel}
\usepackage{amssymb}
\usepackage{amsmath}
\usepackage{graphicx}
\usepackage{float}
\usepackage{shortvrb}
\usepackage{cancel}
\usepackage[T1]{fontenc}
\usepackage{nicefrac}
\usepackage{amsfonts}
%nomenclature
\usepackage{makeidx}
\makeindex
\usepackage{nomencl}
\makenomenclature
\renewcommand{\nomname}{List of Symbols}

\usepackage{standalone} %Makes it possible to ignore other preambles of child document
\usepackage{eurosym} %Euro teken mogelijk
\usepackage{multirow} %For multiple rows togheter in one table
\usepackage{parskip} %For a small white line between paragraphs
\usepackage[protrusion=true,expansion=true]{microtype}
\usepackage{hyperref}%For automatic and URL Reference
\usepackage[titletoc]{appendix}
%Possible to change the margins
\usepackage{geometry}
\geometry{verbose,tmargin=1.9cm,bmargin=1.7cm,lmargin=1cm,rmargin=1cm}

\usepackage{subfig}

%include pdf pages
\usepackage{pdfpages}

%Small titles
\usepackage[small,compact]{titlesec}

%being able to create tables over multiple pages
\usepackage{longtable}

\makeatletter

%Change standard font size
\renewcommand\normalsize{ \@setfontsize\normalsize{11pt}{11pt}}\normalsize  
\makeatother

\usepackage{fancyhdr}
\pagestyle{fancy}
\fancyhead{}
\fancyfoot{}

%Gives text above each page
\fancyhead[CO,CE]{DSE Project}

%Page number
\fancyfoot[RO,LE]{\thepage}


\usepackage{babel}

%Available structures:
%Report: \part{}, \chapter{}, \section{}, \subsection{}, \subsubsection{}, \paragraph{}, \subparagraph{}

\begin{document}
\chapter{Introduction}
The DSE Project to design an all plastic, high altitude observation UAV is performed by ten students of the TU Delft. This project plan gives an insight in the project itself and the planning for the upcoming weeks. First, the project is analysed on different criteria. The Mission Need Statement (MNS) and Project Objective Statement (POS) are given. Top level requirements given are shown, from which a general description of the entire system is derived. Next the project planning and organisation is described.

The planning and organisation is divided into several sections. It starts with a Work Flow Diagram (WFD) which shows different tasks to be performed and the order they need to be performed in. From this, a more elaborated diagram on the work is derived, the Work Breakdown Structure (WBS) is explained. An Organisational Breakdown Structure will show who will perform which task. The above is put together in a Gantt chart, showing the activities on a timeline for the upcoming weeks.

At the end, sustainability is taken into account. Because sustainable development is becoming more and more dominant in designing, the impact of the design on the environment will be monitored. It can play a role in the decision making for different materials or propulsion systems e.g. The approach on how to implement sustainability is given this section.

\nomenclature{BD}{Block diagram}
\nomenclature{BS}{Breakdown structure}
\nomenclature{BR}{Baseline review}
\nomenclature{CS}{Control and stability}
\nomenclature{DSE}{Design synthesis exercise}
\nomenclature{DOS}{Design option structuring}
\nomenclature{FB}{Functional breakdown}
\nomenclature{FFD}{Functional flow diagram}
\nomenclature{HR}{Human resources}
\nomenclature{LS}{Literature study}
\nomenclature{MNS}{Mission need statement}
\nomenclature{MTR}{Midterm review}
\nomenclature{OBS}{Organisational breakdown structure}
\nomenclature{PD\&D}{Project design and development logic}
\nomenclature{PO}{Project objective}
\nomenclature{POS}{Project objective statement}
\nomenclature{P\&P}{Performance and propulsion}
\nomenclature{RDT}{Requirements discovery tree}
\nomenclature{ROI}{Return of investment}
\nomenclature{UAV}{Unmanned aerial vehicle}
\nomenclature{WBS}{Work breakdown structure}
\nomenclature{WFD}{Work flow diagram}
\nomenclature{UAS}{Unmanned aerial system}

\end{document}